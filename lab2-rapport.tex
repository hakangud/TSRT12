\documentclass[a4paper]{article}
\usepackage[swedish]{babel}
\usepackage[utf8]{inputenc}
\usepackage{amsmath}

\title{Labrapport - TSRT12}
\author{Håkan Gudmundsson, D3b \\ Jacob Johansson, D3b}
\date{\today}

\begin{document}

\maketitle
\newpage

\begin{abstract}

Laborationens mål var att bestämma en processmodell för ett system av två vattentankar.
Detta gjordes genom att via mätningar på systemet bestämma tidskonstant och förstärkning.
Nedan visas de uppmätta värdena.
\\\\
\begin{tabular}{l l}
  Tidskonstant: & $T = 22$ \\
  Förstärkning: & $K_{dubbel} = 25$ 
\end{tabular}
\\\\
Som riktlinjer så fanns det några krav som den färdiga regulatorn skulle uppfylla.
Kraven visas nedan.
\\\\
\begin{tabular}{l l}
  Stigtid: & $T_{r} \leq 5s$ \\
  Översläng: & $M \leq 10\%$ \\
  Stationärt fel: & $e_{0} \leq 5\% $ 
\end{tabular}
\\\\
Med hjälp av ovanstående utmätningar och krav kunde följande konstanter beräknas.
\\\\
\begin{tabular}{l}
  $K = 0.52$ \\
  $\beta = 0.13$ \\
  $\tau_{d} = 10.23$ \\
  $\tau_{I} = 37.60$ \\
  $\gamma = 0.6816$ 
\end{tabular}
\\\\
\end{abstract}

\newpage
\tableofcontents
\newpage

\section{Inledning}

Laborationen är den andra laborationen i kursen reglerteknik(TSRT12). \\
Som hjälp för att lösa uppgifterna i laborationen användes förutom kurslitteraturen olika datorprogram. 
\\\\
Två ekvationer ges också ur Lab-PM som beskriver överföringsfunktionen hos en enkeltank samt överföringsfunktionen hos en dubbeltank.
\\
\begin{equation}
  G_{enkel}(s)=\frac{K_{enkel}}{(sT+1)}
\end{equation}
\begin{equation}
  G_{dubbel}(s)=\frac{K_{dubbel}}{(sT+1)^2}
\end{equation}

\subsection{Variabeldefinition}

Variablerna som används i laborationen defineras nedan.
\\\\
\begin{tabular}{l l l}
  $\delta_{u}(t)$ & Avvikelse från arbetspunkt & (V) \\
  $K_{enkel}$ & Proportionalitetskonstant (enkeltank) & ($cm^3/V$) \\
  $K_{dubbel}$ & Proportionalitetskonstant (dubbeltank) & ($cm^3/V$)
\end{tabular}

\section{Bestämma $K_{dubbel}$}

Överföringsfunktion (2) används.
\\
\begin{equation}
  G_{dubbel}(s)=\frac{K_{dubbel}}{(sT+1)^2}
\end{equation}
Insignalen får vara ett steg. Där $c$ är en godtycklig konstant
\\
\begin{equation*}
\delta_{u}(s)=\begin{cases}
  c & \text{t $\geq$ 0}  \\
  0 & \text{t $<$ 0} 
\end{cases}
\end{equation*}

\section{Slutsats}

Det slutgiltiga uttrycket för regulatorn är:
\\
\begin{equation*}
  F(s)=0.52\frac{10.23+1}{1.33s+1} \frac{37.6s+1}{37.6s+0.6816}
\end{equation*}
\\
Några problem som stöttes på under labben var bland annat att vissa av pumparna fungerade dåligt och gav konstiga resultat.
Antagligen så berodde detta på att sensorerna som mätte vattennivån inte fungerade som de skulle.
Hade utrustningen fungerat bättre så hade vi kunna nått ett bättre resultat på mindre tid.
Dock så är ju inte alla system som man kan tänkas stöta på ute i arbetslivet helt perfekta heller. 

\end{document}