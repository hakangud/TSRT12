\documentclass[a4paper]{article}
\usepackage[swedish]{babel}
\usepackage[utf8]{inputenc}
\usepackage{amsmath}

\title{Labrapport - TSRT12}
\author{Håkan Gudmundsson, D3b \\ Jacob Johansson, D3b}
\date{\today}

\begin{document}

\maketitle
\newpage

\begin{abstract}

Laborationens mål var att bestämma en processmodell för ett system av två vattentankar.
Detta gjordes genom att via mätningar på systemet bestämma tidskonstant och förstärkning.
Nedan visas de uppmätta värdena.
\\\\
\begin{tabular}{l l}
  Tidskonstant: & $T = 22$ \\
  Förstärkning: & $K_{dubbel} = 25$ 
\end{tabular}
\\\\
Som riktlinjer så fanns det några krav som den färdiga regulatorn skulle uppfylla.
Kraven visas nedan.
\\\\
\begin{tabular}{l l}
  Stigtid: & $T_{r} \leq 5s$ \\
  Översläng: & $M \leq 10\%$ \\
  Stationärt fel: & $e_{0} \leq 5\% $ 
\end{tabular}
\\\\
Med hjälp av ovanstående utmätningar och krav kunde följande konstanter beräknas.
\\\\
\begin{tabular}{l}
  $K = 0.52$ \\
  $\beta = 0.13$ \\
  $\tau_{d} = 10.23$ \\
  $\tau_{I} = 37.60$ \\
  $\gamma = 0.6816$ 
\end{tabular}
\\\\
\end{abstract}

\newpage
\tableofcontents

\end{document}